\documentclass[12pt]{article}
\usepackage{float}
\usepackage{sbc-template}
\usepackage{graphicx,url}
\usepackage[utf8]{inputenc}
\usepackage[brazil]{babel}
%\usepackage[latin1]{inputenc}  
\raggedbottom
\setlength{\parskip}{0.5pt}
     
\sloppy

\title{Aprendizagem ativa de Engenharia de Software Experimental: Uma análise sobre autoconfiança e aprendizado}
\author{Caio César Sousa Bandeira\inst{1}, Ranya Duran Greco\inst{1}, Carlos Augusto Carneiro \\de Freitas Filho\inst{1}, Júlia Freitas Santos\inst{1}, Anna Beatriz Marques\inst{1}}
\address{Universidade Federal do Ceará - Campus Russas (UFC)
Avenida Felipe Santiago\\411, Campo Federal, Russas - CE  
\email{\{caiocezzar, ranyagreco, carlosaugustocarneiro, juliafreitas\}@alu.ufc.br}
\email{beatriz.marques@ufc.br}}
\begin{document} 

\maketitle

\begin{abstract}
This article reports an experience in teaching Experimental Software Engineering in a 64-hour undergraduate course. The pedagogical approach combined interactive lectures, guided studies, seminars, and practical activities involving statistical and qualitative analysis. A perception survey answered by 21 students assessed learning, motivation, and confidence in conducting empirical studies. Results show increased methodological knowledge, solid performance in the assignments, and a strong influence of motivation on students’ confidence. Psychometric analysis indicates high consistency in the knowledge blocks and the need for adjustments in motivation-related items. The experience contributes evidence on effective practices for teaching.
\end{abstract}
     
\begin{resumo} 
Este artigo apresenta um relato de experiência no ensino de Engenharia de Software Experimental em uma disciplina de 64 horas de graduação. A abordagem combinou aulas dialogadas, estudos dirigidos, seminários e atividades práticas de análise estatística e qualitativa. Uma pesquisa de percepção respondida por 21 estudantes avaliou aprendizagem, motivação e autoconfiança. Os resultados mostram aumento na percepção do nível de conhecimento, bom desempenho nas atividades e influência da motivação na confiança discente. A análise psicométrica do questionário utilizado indica alta consistência nos blocos de conhecimento e necessidade de ajustes nos itens de motivação. A experiência contribui com evidências sobre práticas eficazes para o ensino.
\end{resumo}


\section{Introdução}

A Engenharia de Software Experimental (ESE) é reconhecida como um elemento central para o amadurecimento da Engenharia de Software enquanto área científica, pois possibilita que decisões técnicas sejam fundamentadas em evidências empíricas, e não apenas em intuição ou experiência isolada \cite{mendez2024handbook}. O domínio de métodos empíricos permite aos profissionais compreenderem de forma mais crítica os benefícios e limitações de tecnologias, processos e ferramentas, contribuindo para a tomada de decisão baseada em dados na prática industrial \cite{avgeriou2024designing}.

Apesar de esforços recentes da literatura em estruturar o ensino de ESE e aproximar os estudantes dos fundamentos da pesquisa empírica \cite{mendez2024handbook, avgeriou2024designing}, a formação nessas competências na graduação ainda representa um desafio. Integrar o rigor científico ao desenvolvimento de software exige abordagens pedagógicas que vão além das aulas expositivas tradicionais, demandando a vivência prática de métodos experimentais e investigativos \cite{irabedra2025active}.

Nesse contexto, diversos estudos apontam que currículos convencionais frequentemente falham em engajar os alunos na complexidade inerente à condução de experimentos, o que contribui para um distanciamento entre teoria e prática \cite{mendez2024handbook, irabedra2025active}. Como alternativa, estratégias de Aprendizagem Ativa (\textit{Active Learning}) têm ganhado destaque na educação em computação, ao reposicionar o estudante como agente central do processo de aprendizagem \cite{cordova2024active}. Revisões sistemáticas indicam que essas abordagens favorecem maior engajamento, melhor retenção de conteúdo e desenvolvimento do pensamento crítico, superando métodos puramente expositivos \cite{cordova2024active}.

Na Engenharia de Software, estratégias como atividades práticas, aprendizagem baseada em projetos e aprendizagem baseada em exemplos demonstram potencial para aumentar a motivação discente e facilitar a compreensão de conceitos abstratos ou complexos \cite{irabedra2025active, bonetti2025example}. No entanto, embora o propósito dessas abordagens seja justamente ir além do conhecimento técnico para desenvolver competências interpessoais e de resolução de problemas \cite{cordova2024active}, ainda se observa que muitos estudos privilegiam métricas de desempenho acadêmico, como notas e aprovação, enquanto os aspectos subjetivos do processo de aprendizagem recebem atenção mais limitada na literatura.

Em nosso trabalho anterior \cite{greco2025learning}, apresentamos a estruturação de uma disciplina de ESE baseada em Aprendizagem Ativa e validamos sua eficácia quanto ao engajamento, à percepção geral de aprendizado e motivação. No entanto, enquanto aquela análise concentrou-se majoritariamente nos resultados acadêmicos e na aceitação da metodologia, a literatura ainda carece de investigações focadas na evolução da confiança do estudante. Este artigo difere dessa abordagem ao deslocar o foco da validação do método para a transformação do indivíduo: investigamos especificamente a autoeficácia percebida, analisando se a vivência prática constrói a segurança necessária para que o discente se sinta apto a conduzir experimentos na vida profissional, para além das métricas tradicionais de aprovação. 

\section{Fundamentação Teórica} \label{sec:firstpage}

\subsection{Engenharia de Software Experimental}

Historicamente, a Engenharia de Software caracterizou-se por uma forte dependência do julgamento humano e da experiência individual, com a adoção de tecnologias e metodologias frequentemente orientada por tendências ou preferências subjetivas. Esse cenário contribuiu para a consolidação de práticas pouco fundamentadas empiricamente, dificultando a previsibilidade e a garantia da qualidade nos projetos de software.

Nesse contexto, críticas como as de \cite{tichy1998should} questionaram a ausência de rigor científico na área, ao apontar a escassez de dados empíricos que sustentassem muitas das afirmações feitas na Ciência da Computação. Em resposta a essas limitações, emergiu a ESE, cujo objetivo central é promover a tomada de decisão baseada em evidências, por meio da aplicação sistemática de métodos científicos, como experimentos, estudos de caso e surveys. Conforme argumenta \cite{basili1999building}, a experimentação é essencial para consolidar a Engenharia de Software como uma disciplina científica, permitindo a avaliação objetiva e reprodutível de técnicas, processos e ferramentas.

No âmbito educacional, a ESE pressupõe a integração entre teoria e prática, possibilitando que o estudante planeje, execute e analise estudos empíricos de forma estruturada. Alinhada a essa perspectiva, a abordagem proposta por \cite{greco2025learning} incorpora a execução de estudos primários no ensino de graduação, aproximando o discente das práticas adotadas em uma Engenharia de Software orientada por evidências.

\subsection{Estratégias de pesquisa empírica}
A seleção da estratégia de pesquisa adequada é guiada pelo tipo de questão que se deseja responder e pelo nível de controle sobre o ambiente. Conforme categorizado por \cite{easterbrook} e discutido por \cite{travassos2002introducao}, as abordagens empíricas na Engenharia de Software dividem-se primariamente em:

\begin{itemize}
    \item \textit{Surveys}: É uma estratégia de pesquisa, frequentemente retrospectiva, utilizada para coletar dados qualitativos ou quantitativos de uma amostra representativa. Seu objetivo principal é descrever características, opiniões ou comportamentos de um grande grupo de indivíduos, permitindo generalizações estatísticas sobre a população-alvo.
    \item \textit{Case Studies}: Segundo \cite{travassos2002introducao}, são investigações empíricas que analisam um fenômeno contemporâneo dentro do seu contexto real. Essa abordagem é especialmente útil quando os limites entre o fenômeno e o contexto não são claramente evidentes e não é possível isolar o comportamento do sistema do seu ambiente operacional.
    \item \textit{Experiments}: São investigações rigorosas onde uma ou mais variáveis independentes são manipuladas para medir seu efeito sobre variáveis dependentes, mantendo as demais constantes. Essa estratégia oferece o maior nível de controle e é a única capaz de provar relações de causalidade (causa e efeito) com alta precisão estatística."
\end{itemize}

\section{Trabalhos Relacionados}

A literatura recente sobre o ensino de Engenharia de Software tem destacado a transição de modelos tradicionais para abordagens baseadas em aprendizagem ativa. \cite{cordova2024active}, em uma revisão sistemática, identificaram que estratégias centradas no estudante favorecem não apenas a retenção de conteúdo, mas também o desenvolvimento do pensamento crítico. No mesmo sentido, \cite{irabedra2025active} e \cite{bonetti2025example} demonstram que o uso de projetos práticos e exemplos reais aumenta o engajamento discente, mitigando a abstração excessiva de conceitos técnicos.
ESE, \cite{mendez2024handbook} e \cite{avgeriou2024designing} discutem a importância de estruturar currículos que combinem teoria estatística com a execução prática, preparando o estudante para uma indústria orientada por evidências.

Trabalhos recentes têm investigado a aplicação dessas estratégias no ensino de ESE. \cite{meireles2024experience} analisaram a percepção de estudantes de graduação e pós-graduação sobre o uso de aprendizagem ativa, focando majoritariamente em atividades de Mapeamento Sistemático. Inspirados por essa abordagem, \cite{greco2025learning} adaptaram a metodologia especificamente para o contexto de graduação, deslocando o foco para a execução de estudos primários, como experimentos, estudos de caso e surveys. Em sua análise, \cite{greco2025learning} validaram a eficácia da abordagem pedagógica, demonstrando que 88,2\% dos estudantes relataram maior segurança ao final do curso e que a motivação apresentou correlação com a percepção de valor da disciplina.

Em contraste com esses estudos, este trabalho, embora situado no mesmo contexto de ensino prático explorado por \cite{greco2025learning}, avança a investigação ao deslocar o foco da validação pedagógica geral para uma análise psicométrica da autoeficácia percebida, entendida como a crença do estudante em sua capacidade de planejar, conduzir e analisar estudos empíricos. Enquanto os trabalhos anteriores avaliaram predominantemente satisfação e desempenho, esta pesquisa investiga estatisticamente como a motivação em tópicos específicos, como análise qualitativa e estatística, influencia a construção da confiança discente. Diferentemente de uma replicação direta, este estudo utiliza o mesmo contexto pedagógico como base para investigar um fenômeno distinto, buscando compreender os fatores que levam o estudante a se sentir apto a aplicar ESE na prática profissional.

\section{Metodologia}

Esse trabalho descreve uma experiência de ensino em uma turma da disciplina eletiva de ESE de 64 horas, ministrada no campus de uma universidade, para cursos de graduação. O objetivo geral da disciplina é capacitar o estudante nos fundamentos de ESE, abrangendo estudos primários e secundários. %A ementa inclui: conceituação e esclarecimento sobre experimentos controlados, estudos de caso e surveys; o processo de desenvolvimento de um projeto de pesquisa (incluindo atividades, formulação de questões de pesquisa, construção de teoria e análise de dados qualitativa/quantitativa); investigação de experimentos em engenharia de software; e prática supervisionada através de um experimento de engenharia de software em pequena escala.%

Durante a disciplina, a professora responsável e três monitores realizaram o acompanhamento das atividades. A turma foi composta por 63 estudantes, sendo 39 do curso de Engenharia de Software e 24 de Ciência da Computação. A turma foi conduzida através de uma combinação de estratégias: aulas dialogadas, estudos dirigidos, seminários, exercícios em duplas ou grupos e trabalhos práticos colaborativos. Cada conjunto de conteúdos foi tratado segundo a seguinte organização.

Para os estudos primários, a professora realizou aulas interativas fundamentadas em \cite{wohlin2012experimentation}, destacando conceitos essenciais, o papel da experimentação em Engenharia de Software e diferentes tipos de estudos primários (como experimentos controlados, estudos de caso, pesquisa-ação e pesquisas de opinião). Em paralelo, exemplos selecionados de estudos primários foram incluídos nas aulas para ilustrar a aplicação prática desses conceitos.

Nos estudos secundários, a docente adotou aulas expositivas interativas, baseadas em trabalhos clássicos sobre revisões sistemáticas, mapeamentos sistemáticos e revisões terciárias, como os de \cite{kitchenham2004procedures}. Nessa abordagem, exemplos de revisões publicadas foram discutidos brevemente em sala.

O tópico experimentos controlados foi aprofundado por meio de estudos dirigidos em grupos de até 3 alunos. Os estudantes foram orientados a ler e sintetizar os capítulos 6, 7, 8 e 9 de \cite{wohlin2012experimentation} que tratam de processo, escopo, planejamento e execução do experimento. A partir dessas leituras, cada grupo preparou e apresentou seminários para a turma, utilizando os materiais produzidos nos estudos dirigidos.

Para a análise estatística de experimentos controlados, foi desenvolvido uma aula expositiva interativa abordando estatística descritiva, tipos e escalas de variáveis, verificação de normalidade e testes de hipóteses. Em seguida, organizou-se uma sessão prática de laboratório conduzida pelos monitores usando o software JASP\footnote{JASP: https://jasp-stats.org/} (software open-source com interface
gráfica que integra vários testes estatísticos), na qual se seguiu um roteiro estruturado de análise: preparação da base de dados do experimento, análise estatística descritiva, teste de normalidade e testes de hipóteses. Nesse contexto, utilizaram-se dados do experimento controlado publicado por \cite{fonseca2024avaliaccao} na atividade prática.

O trabalho prático 1 foi planejado com o objetivo de consolidar a compreensão de experimentos controlados e da respectiva análise estatística. Para isso, foi formado grupos de três estudantes, responsáveis por analisar um artigo científico que descrevia um experimento controlado, extrair seus elementos metodológicos centrais (objetivos, contexto, variáveis, tratamentos e hipóteses) e realizar uma análise estatística completa, incluindo estatística descritiva, teste de normalidade e testes de hipóteses. Posteriormente, cada grupo apresentou em sala suas principais interpretações e resultados.

No que se refere aos estudos de caso, os estudantes realizaram um novo estudo dirigido, sobre o Capítulo 5 de \cite{wohlin2012experimentation}, que examina a metodologia de estudo de caso em Engenharia de Software.

O tópico de pesquisas de opinião pessoal (\textit{Surveys}) foi trabalhado por meio de outro estudo dirigido baseado no Capítulo 3 de \cite{shull2008guide}, que discute implementação e dificuldades típicas de \textit{surveys} em contextos de Engenharia de Software.

A análise qualitativa foi introduzida com uma aula expositiva dialogada baseada em materiais do minicurso de \cite{martinelli2023analise}, abordando técnicas como codificação aberta, codificação fechada e \textit{Grounded Theory}. Os estudantes participaram de uma atividade prática em sala dedicada à codificação aberta de dados. Em laboratório, os monitores orientaram uma sessão voltada à codificação fechada e à visualização de resultados, utilizando a ferramenta Taguette\footnote{Taguette: https://www.taguette.org/} para apoio à análise de conteúdo e a plataforma  Flourish\footnote{Flourish: https://flourish.studio/} para a construção de visualizações de dados qualitativos. Para tornar a atividade mais concreta, foi utilizado uma parte da amostra de dados qualitativos provenientes do estudo de \cite{desiderio2024ready}.

O segundo trabalho prático teve como foco um estudo empírico. Foram formados grupos de até seis estudantes, e a cada grupo foi atribuído um tipo de estudo primário (experimento controlado, estudo de caso ou \textit{survey}), atribuídos aleatoriamente (três equipes por método). Coube aos grupos planejar o estudo (incluindo objetivos, variáveis, participantes, contexto, procedimentos e instrumentos de coleta), executar a investigação e realizar a análise dos dados, seja quantitativa, seja qualitativa. A atividade resultou em duas apresentações: uma de planejamento, voltada ao delineamento do estudo, e outra de resultados, contemplando procedimentos de execução, análises realizadas e reflexões sobre o aprendizado proporcionado.

A metodologia adotada foi inspirada na abordagem descrita por \cite{meireles2024experience}, porém ajustada às características do perfil de estudantes de gradução da turma:
\begin{itemize}
    \item Aulas introdutórias sobre estudos primários e secundários alinhadas ao nível da turma;
    \item Sessões práticas de laboratório em análise estatística com apoio de ferramentas como JASP\footnote{JASP: https://jasp-stats.org/};
    \item A inclusão sistemática de análise qualitativa com uso de Taguette\footnote{Taguette: https://www.taguette.org/} e  Flourish\footnote{Flourish: https://flourish.studio/}. 
    \item Em função de limitações de espaço físico e do tamanho da turma, debates em grande grupo não foram priorizados;
    \item O tema Design Science Research não foi abordado uma vez que a disciplina não tinha como público-alvo estudantes de pós-graduação.
    \item Diferentemente do que é descrito nos trabalhos práticos de \cite{meireles2024experience}, que enfatizava etapas de mapeamento sistemático e o planejamento de novos experimentos, nesta experiência concentrou-se o esforço em atividades práticas com estudos primários, mantendo a reanálise de um experimento controlado e adicionando a execução completa de um estudo primário (experimento, estudo de caso ou \textit{survey}) como eixo central de aprendizagem.
    
\end{itemize}

\subsection{Pesquisa de Percepção dos Estudantes}
A percepção dos estudantes sobre a disciplina foi investigada por meio de uma pesquisa de opinião, estruturada a partir do instrumento adotado por \cite{meireles2024experience}. O questionário online disponibilizado na plataforma Google Forms, composto por 31 questões (24 fechadas e 7 abertas) distribuídas em cinco blocos temáticos: dados demográficos, percepção de aprendizagem antes da disciplina, percepção de aprendizagem após a disciplina, motivação para aprender e percepção de valor atribuído à aprendizagem. O questionário completo, bem como a descrição detalhada dos procedimentos de análise empregados para cada variável estão disponíveis no repositório Github.

A participação na pesquisa foi voluntária: nenhum estudante foi obrigado a responder ao questionário, nem sofreu qualquer tipo de penalidade em caso de desistência. Os participantes receberam o Termo de Consentimento Livre e Esclarecido (TCLE) e só puderam prosseguir após registrar concordância. Esse termo ressaltava a confidencialidade dos dados, a inexistência de riscos envolvidos e o uso exclusivo das respostas para fins acadêmicos, assegurando o anonimato e a não divulgação de informações identificáveis em publicações.

A divulgação do questionário foi realizada em sala de aula, ao final da disciplina, após as apresentações dos trabalhos das equipes. Os estudantes tiveram uma semana para responder ao instrumento. A pesquisa considerou exclusivamente discentes matriculados na disciplina de Engenharia de Software Experimental no semestre analisado, totalizando 63 estudantes, dos quais 21 responderam ao questionário. A coleta ocorreu após a conclusão de todas as atividades previstas na disciplina.

Para o tratamento dos dados, foi adotada uma abordagem exclusivamente quantitativa. Os dados foram organizados, analisados e visualizados por meio de notebooks Python, com o uso de bibliotecas amplamente empregadas em análise de dados e estatística, como Pandas\footnote{Pandas: https://pandas.pydata.org/}
 para manipulação e organização dos dados, Scipy\footnote{Scipy: https://scipy.org/}
 para análises estatísticas e Matplotlib\footnote{Matplotlib: https://matplotlib.org/}
 para a geração dos gráficos. A implementação e execução dos scripts ocorreram na IDE Cursor\footnote{Cursor: https://www.cursor.so/}
, utilizando notebooks Jupyter\footnote{Jupyter: https://jupyter.org/}
, que permitiram a execução interativa das análises e a visualização dos resultados.

A análise contemplou estatísticas descritivas e visualizações por meio de boxplots, com o objetivo de examinar o desempenho dos estudantes nas atividades propostas. Adicionalmente, foram conduzidas análises de regressão para investigar a relação entre variáveis associadas à motivação, conhecimento prévio e confiança percebida, bem como análises de confiabilidade para avaliar a consistência interna dos blocos do instrumento de coleta. Os procedimentos estatísticos foram realizados de forma colaborativa por dois pesquisadores, visando maior rigor na condução e interpretação dos resultados.

\section{Resultados}
Esta seção apresenta os resultados obtidos a partir dos dados coletados por meio do questionário e das planilhas de acompanhamento de notas/atividades. Os dados quantitativos são apresentados utilizando representações gráficas e análises estatísticas 

\subsection{Desempenho da turma}
A Figura \ref{fig:boxplotsNotas} apresenta a distribuição das notas dos trabalhos práticos TP1 e TP2. Em ambos, a maior parte das notas concentra-se entre 7,5 e 9,5, indicando desempenho geral elevado da turma.

Observa-se que a mediana do TP2 é ligeiramente superior à do TP1, sugerindo melhora no desempenho ao longo da disciplina. O TP1 apresenta maior variabilidade, enquanto o TP2 mostra distribuição mais concentrada, com menor dispersão. Em ambos os trabalhos, há outliers associados a notas muito baixas, representando casos isolados de baixo desempenho.

De modo geral, os resultados indicam uma progressão no desempenho entre as duas avaliações e um nível de desempenho consistente para a maioria dos estudantes.

\begin{figure}[H]
    \centering
    \includegraphics[width=0.5\linewidth]{imgs/boxplot_notas.png}
    \caption{Performance dos estudantes nos trabalho práticos}
    \label{fig:boxplotsNotas}
\end{figure}

\subsection{Percepção de motivação}
A Figura \ref{fig:boxplots} apresenta uma análise comparativa do nível de conhecimento dos estudantes em sete tópicos de metodologia de pesquisa experimental, avaliados antes e depois da disciplina, por meio de gráficos de violino com boxplots internos e pontos individuais sobrepostos.

Cada subgráfico corresponde a um tópico específico — Estudos Experimentais, Experimento Controlado, Estudo de Caso, Survey, Análise Estatística, Análise Qualitativa e Revisão Sistemática da Literatura (RSL) — comparando as condições \textbf{“Antes”} e \textbf{“Depois”}. A escala de resposta varia de 0 a 5, representando o nível de conhecimento autoavaliado.

A visualização permite observar mudanças nas medianas, na dispersão e na forma das distribuições entre as duas condições, bem como a presença de valores atípicos, possibilitando uma avaliação visual do impacto da disciplina sobre o conhecimento dos estudantes em cada tópico.

\begin{figure}[H]
    \centering
    \includegraphics[width=1\linewidth]{imgs/boxplots.png}
    \caption{Distribuição dos niveis de conhecimento antes e depois do curso através de sete tópicos}
    \label{fig:boxplots}
\end{figure}

\subsection{Análise de Regressão}
Para identificar quais fatores influenciam a confiança dos estudantes em conduzir pesquisas experimentais, foi ajustado um modelo de \textbf{Random Forest Regressor}. Esse modelo é particularmente adequado quando há múltiplas variáveis preditoras potencialmente correlacionadas, pois captura relações não lineares e interações complexas entre os atributos.

A importância das variáveis foi calculada com base na redução média de impureza, refletindo a contribuição relativa de cada preditor para a capacidade do modelo em explicar a variância da variável dependente (confiança do estudante). Os valores foram normalizados e apresentados em porcentagem para facilitar a interpretação comparativa.

A Figura \ref{fig:importanciaIndividual} apresenta a importância relativa das variáveis no modelo, evidenciando que \textbf{a motivação média dos estudantes} é, de longe, o fator mais determinante da confiança, representando \textbf{41,1\%} da contribuição total. Esse resultado sugere que a percepção de motivação ao longo da disciplina está positivamente associada à autoconfiança para aplicar conceitos de pesquisa experimental.

O segundo fator mais relevante é a \textbf{motivação relacionada à Análise Qualitativa} (20,1\%), indicando que áreas específicas do conteúdo também podem influenciar substancialmente o sentimento de competência dos estudantes.

Em seguida, surgem três variáveis com contribuições semelhantes:
\begin{itemize}
    \item \textbf{Engajamento} (8,4\%)
    \item \textbf{Conhecimento prévio} (8,1\%)
    \item \textbf{Motivação para Survey} (8,0\%)
\end{itemize}
Esses resultados apontam que tanto fatores comportamentais (engajamento nas atividades) quanto fatores cognitivos (base de conhecimento anterior) desempenham papel moderado, porém relevante, na formação da confiança. A motivação por tópicos específicos, como Estudo de Caso (6,7\%) e RSL (4,7\%), também possui contribuição, embora menor.

As variáveis de menor impacto foram \textbf{motivação para análise estatística} (1,9\%) e o \textbf{curso de origem} do estudante (0,9\%). Isso sugere que a confiança adquirida ao longo da disciplina é mais fortemente influenciada por aspectos motivacionais e pedagógicos do que por características demográficas ou estruturais.

 \begin{figure}[!ht]
     \centering
     \includegraphics[width=0.75\linewidth]{imgs/importancia_individual.png}
     \caption{Importância das variáveis na confiança do estudante}
     \label{fig:importanciaIndividual}
 \end{figure}

Ao agrupar as variáveis preditoras do modelo por categorias funcionais, observa-se que os aspectos motivacionais — englobando tanto a motivação geral quanto a específica por tópicos — são os principais determinantes da confiança discente, respondendo por 82,6\% da importância total. Este resultado sugere que estudantes que se percebem mais motivados durante o curso tendem a relatar níveis significativamente mais elevados de autoconfiança para a aplicação prática dos conceitos de ESE, superando o peso de fatores comportamentais e cognitivos. Em uma posição secundária, as categorias de Engajamento (8,4\%) e Conhecimento Prévio (8,1\%) contribuem de forma moderada para o modelo, enquanto o curso de origem (0,9\%) apresenta impacto quase nulo, indicando que a confiança adquirida não depende da formação base (Engenharia de Software ou Ciência da Computação) do estudante.

\subsection{Análise de Confiabilidade e Consistência}
Para verificar a robustez do instrumento de coleta de dados e garantir que as percepções relatadas pelos estudantes fossem consistentes, realizou-se uma análise psicométrica baseada em três indicadores principais:

\begin{itemize}
    \item \textbf{Consistência Interna por Bloco ($\alpha$ de Cronbach):} Esta métrica avalia o quão bem um conjunto de itens (perguntas) mede um mesmo conceito ou dimensão (ex: Bloco de Conhecimento, Bloco de Motivação). Os blocos foram definidos seguindo as dimensões originais do instrumento de Meireles et al. (2024). Um valor de $\alpha$ acima de 0,70 é geralmente considerado aceitável, indicando que as perguntas de um mesmo bloco são compreendidas de forma coesa pelos respondentes.
    \item \textbf{Correção Item-Total Correlacionada:} Este índice verifica a correlação de cada pergunta individual com o escore total do seu respectivo bloco. Em termos práticos, ele indica se um item específico ``conversa'' bem com os demais ou se ele está medindo algo diferente do pretendido. Valores abaixo de 0,30 sugerem que o item pode ser ambíguo.
    \item \textbf{$\alpha$ de Cronbach se o Item for Deletado:} Esta análise simula o que aconteceria com a confiabilidade do bloco caso uma pergunta específica fosse removida. Se a exclusão de um item aumenta significativamente o $\alpha$ total, isso sinaliza que aquele item estava ``puxando a consistência para baixo'' e deve ser revisado em futuras aplicações do curso.
\end{itemize}
\subsubsection{Consistência interna por bloco}
A Figura \ref{fig:AlphaAnalisys} apresenta os valores de Alpha de Cronbach para os três blocos multidimensionais analisados. O Bloco 1 (Conhecimento Antes) exibiu consistência interna excelente ($\alpha$
 = 0.928), enquanto o Bloco 2 (Conhecimento Depois) apresentou consistência boa ($\alpha$
 = 0.832), ambos acima dos limites recomendados para instrumentos educacionais ($\alpha$ $\geq$
 0.70). Em contraste, o Bloco 3 (Motivação) obteve $\alpha$
 = 0.616, valor classificado como questionável, indicando menor homogeneidade entre os itens.
\begin{figure}[H]
    \centering
    \includegraphics[width=1\linewidth]{cronbach_alpha_analysis.png}
    \caption{Alpha de cronbach e correlações}
    \label{fig:AlphaAnalisys}
\end{figure}
\subsubsection{Correlações Item-total}
As Figuras \ref{fig:AlphaAnalisys}b, \ref{fig:AlphaAnalisys}c e \ref{fig:AlphaAnalisys}d apresentam as correlações item-total para os três blocos.
No Bloco 1, todas as correlações situaram-se entre 0.555 e 0.888, significativamente acima do mínimo aceitável (0.30), demonstrando forte convergência dos itens para o constructo medido. O Bloco 2 mostrou correlações entre 0.494 e 0.740, indicando relação consistente, ainda que ligeiramente menos robusta que no Bloco 1. Já no Bloco 3, as correlações variaram entre 0.323 e 0.472, valores próximos ao limiar mínimo, sugerindo que os itens compartilham variância, mas de forma menos expressiva. 

\subsubsection{Avaliação de Alpha se Item Deletado}
A Figura \ref{fig:alphaSeDeletado}a demonstra que, no Bloco 1, a exclusão de qualquer item produziria aumento marginal no valor de $\alpha$  (máximo de +0.008), indicando que todos contribuem adequadamente para a consistência interna. Situação semelhante é observada na Figura \ref{fig:alphaSeDeletado}b para o Bloco 2, onde a variação máxima ao remover um item foi pequena (±0.049), não comprometendo a coerência do bloco.

No Bloco 3 (Figura \ref{fig:alphaSeDeletado}c), a análise revela que a exclusão do item relacionado à Revisão Sistemática da Literatura (RSL) resulta em um aumento do alpha para 0.640, sugerindo que este item apresenta menor aderência ao constructo de motivação. Em contrapartida, a exclusão do item “Estudo de Caso” reduz substancialmente o $\alpha$
 (para 0.520), indicando que este item possui maior contribuição para a consistência interna do bloco.

\begin{figure}
    \centering
    \includegraphics[width=1\linewidth]{cronbach_alpha_if_deleted.png}
    \caption{Alpha de Cronbach se item deletado}
    \label{fig:alphaSeDeletado}
\end{figure}
No conjunto, os resultados indicam que os Blocos 1 e 2 possuem estrutura psicométrica sólida, com altos níveis de consistência interna e forte alinhamento entre os itens. Já o Bloco 3 apresenta indícios de fragilidade estrutural, evidenciados pelo alpha reduzido, correlações item-total próximas ao limiar de aceitabilidade e impacto positivo da remoção de itens específicos. Tais evidências sugerem a necessidade de revisão ou expansão do conjunto de itens relacionados à motivação para aprimoramento de sua confiabilidade em futuras aplicações do instrumento.

\section{Considerações finais}
Os resultados indicam que estratégias de aprendizagem ativa contribuem para o desenvolvimento do conhecimento metodológico e para o fortalecimento da autoconfiança dos estudantes em Engenharia de Software Experimental, sendo a motivação o fator mais fortemente associado à confiança para conduzir pesquisas empíricas. Esse achado reforça a relevância de abordagens pedagógicas que promovam envolvimento ativo e significado percebido ao longo do processo de aprendizagem.

Como perspectivas de trabalhos futuros, destacam-se o aprimoramento do instrumento de coleta, especialmente com a inclusão de novos itens para melhor captar a percepção de contribuição da disciplina e aumentar a confiabilidade do bloco de motivação, bem como a replicação do estudo em outras turmas e contextos institucionais. Além disso, investigações com amostras maiores e análises longitudinais podem oferecer evidências mais robustas sobre o impacto dessas estratégias na formação de profissionais orientados por evidências.

\section*{Artefatos relacionados ao trabalho}
\label{sec:artefatos}
Os artefatos associados a este estudo, incluindo os instrumentos de coleta, os dados coletados e os \textit{notebooks} utilizados nas análises, estão disponíveis publicamente em um repositório da Anonymous Github: \url{https://anonymous.4open.science/r/ees-feedback-F30B}.

\section*{Agradecimentos}

\subsection*{Uso de Inteligência Artificial}
 Neste trabalho, foram utilizadas ferramentas de Inteligência Artificial Generativa exclusivamente para suporte técnico na elaboração de gráficos incluídos na seção de resultados. Para isso, empregou-se o Cursor AI (agente assistente de código), utilizado apenas para auxiliar na depuração do código empregado para produzir os gráficos apresentados no artigo.

Nenhuma ferramenta de IA foi utilizada para redigir, revisar ou estruturar o texto do artigo, tampouco para elaborar citações, interpretações ou discussões. As decisões metodológicas, análises, resultados e conclusões são integralmente de responsabilidade das pessoas autoras.

\bibliographystyle{sbc}
\bibliography{sbc-template}

\end{document}
