\documentclass[12pt]{article}
\usepackage{float}
\usepackage{sbc-template}
\usepackage{graphicx,url}
\usepackage[utf8]{inputenc}
\usepackage[brazil]{babel}
%\usepackage[latin1]{inputenc}  
\raggedbottom
\setlength{\parskip}{1pt}
     
\sloppy

\title{Aprendizagem ativa de Engenharia de Software Experimental: Uma análise sobre autoconfiança e aprendizado}
\author{Caio César Sousa Bandeira\inst{1}, Ranya Duran Greco\inst{1}, Carlos Augusto Carneiro \\de Freitas Filho\inst{1}, Júlia Freitas Santos\inst{1}, Anna Beatriz Marques\inst{1}}
\address{Universidade Federal do Ceará - Campus Russas (UFC)
Avenida Felipe Santiago\\411, Campo Federal, Russas - CE  
\email{\{caiocezzar, ranyagreco, carlosaugustocarneiro, juliafreitas\}@alu.ufc.br}
\email{beatriz.marques@ufc.br}}
\begin{document} 

\maketitle

\begin{abstract}
This article reports an experience in teaching Experimental Software Engineering in a 64-hour undergraduate course. The pedagogical approach combined interactive lectures, guided studies, seminars, and practical activities involving statistical and qualitative analysis. A perception survey answered by 21 students assessed learning, motivation, and confidence in conducting empirical studies. Results show increased methodological knowledge, solid performance in the assignments, and a strong influence of motivation on students’ confidence. Psychometric analysis indicates high consistency in the knowledge blocks and the need for adjustments in motivation-related items. The experience contributes evidence on effective practices for teaching.
\end{abstract}
     
\begin{resumo} 
Este artigo apresenta um relato de experiência no ensino de Engenharia de Software Experimental em uma disciplina de 64 horas de graduação. A abordagem combinou aulas dialogadas, estudos dirigidos, seminários e atividades práticas de análise estatística e qualitativa. Uma pesquisa de percepção respondida por 21 estudantes avaliou aprendizagem, motivação e autoconfiança. Os resultados mostram aumento na percepção do nível de conhecimento, bom desempenho nas atividades e influência da motivação na confiança discente. A análise psicométrica do questionário utilizado indica alta consistência nos blocos de conhecimento e necessidade de ajustes nos itens de motivação. A experiência contribui com evidências sobre práticas eficazes para o ensino.
\end{resumo}


\section{Introdução}

A Engenharia de Software Experimental (ESE) é reconhecida como um elemento central para o amadurecimento da Engenharia de Software enquanto área científica, pois possibilita que decisões técnicas sejam fundamentadas em evidências empíricas, e não apenas em intuição ou experiência isolada \cite{mendez2024handbook}. O domínio de métodos empíricos permite aos profissionais compreenderem de forma mais crítica os benefícios e limitações de tecnologias, processos e ferramentas, contribuindo para a tomada de decisão baseada em dados na prática industrial \cite{avgeriou2024designing}.

Apesar de esforços recentes da literatura em estruturar o ensino de ESE e aproximar os estudantes dos fundamentos da pesquisa empírica \cite{mendez2024handbook, avgeriou2024designing}, a formação nessas competências na graduação ainda representa um desafio. Integrar o rigor científico ao desenvolvimento de software exige abordagens pedagógicas que vão além das aulas expositivas tradicionais, demandando a vivência prática de métodos experimentais e investigativos \cite{irabedra2025active}.

Nesse contexto, diversos estudos apontam que currículos convencionais frequentemente falham em engajar os alunos na complexidade inerente à condução de experimentos, o que contribui para um distanciamento entre teoria e prática \cite{mendez2024handbook, irabedra2025active}. Como alternativa, estratégias de Aprendizagem Ativa (\textit{Active Learning}) têm ganhado destaque na educação em computação, ao reposicionar o estudante como agente central do processo de aprendizagem \cite{cordova2024active}. Revisões sistemáticas indicam que essas abordagens favorecem maior engajamento, melhor retenção de conteúdo e desenvolvimento do pensamento crítico, superando métodos puramente expositivos \cite{cordova2024active}.

Na Engenharia de Software, estratégias como atividades práticas, aprendizagem baseada em projetos e aprendizagem baseada em exemplos demonstram potencial para aumentar a motivação discente e facilitar a compreensão de conceitos abstratos ou complexos \cite{irabedra2025active, bonetti2025example}. No entanto, embora o propósito dessas abordagens seja justamente ir além do conhecimento técnico para desenvolver competências interpessoais e de resolução de problemas \cite{cordova2024active}, ainda se observa que muitos estudos privilegiam métricas de desempenho acadêmico, como notas e aprovação, enquanto os aspectos subjetivos do processo de aprendizagem recebem atenção mais limitada na literatura.

Em nosso trabalho anterior \cite{greco2025learning}, apresentamos a estruturação de uma disciplina de ESE baseada em Aprendizagem Ativa e validamos sua eficácia quanto ao engajamento, à percepção geral de aprendizado e motivação. No entanto, enquanto aquela análise concentrou-se majoritariamente nos resultados acadêmicos e na aceitação da metodologia, a literatura ainda carece de investigações focadas na evolução da confiança do estudante. Este artigo difere dessa abordagem ao deslocar o foco da validação do método para a transformação do indivíduo: investigamos especificamente a autoeficácia percebida, analisando se a vivência prática constrói a segurança necessária para que o discente se sinta apto a conduzir experimentos na vida profissional, para além das métricas tradicionais de aprovação. 

\section{Fundamentação Teórica} \label{sec:firstpage}

\subsection{Engenharia de Software Experimental}

Historicamente, a Engenharia de Software caracterizou-se por uma forte dependência do julgamento humano e da experiência individual, com a adoção de tecnologias e metodologias frequentemente orientada por tendências ou preferências subjetivas. Esse cenário contribuiu para a consolidação de práticas pouco fundamentadas empiricamente, dificultando a previsibilidade e a garantia da qualidade nos projetos de software.

Nesse contexto, críticas como as de \cite{tichy1998should} questionaram a ausência de rigor científico na área, ao apontar a escassez de dados empíricos que sustentassem muitas das afirmações feitas na Ciência da Computação. Em resposta a essas limitações, emergiu a ESE, cujo objetivo central é promover a tomada de decisão baseada em evidências, por meio da aplicação sistemática de métodos científicos, como experimentos, estudos de caso e surveys. Conforme argumenta \cite{basili1999building}, a experimentação é essencial para consolidar a Engenharia de Software como uma disciplina científica, permitindo a avaliação objetiva e reprodutível de técnicas, processos e ferramentas.

No âmbito educacional, a ESE pressupõe a integração entre teoria e prática, possibilitando que o estudante planeje, execute e analise estudos empíricos de forma estruturada. Alinhada a essa perspectiva, a abordagem proposta por \cite{greco2025learning} incorpora a execução de estudos primários no ensino de graduação, aproximando o discente das práticas adotadas em uma Engenharia de Software orientada por evidências.

\section{Trabalhos Relacionados}

A literatura recente sobre o ensino de Engenharia de Software tem destacado a transição de modelos tradicionais para abordagens baseadas em aprendizagem ativa. \cite{cordova2024active}, em uma revisão sistemática, identificaram que estratégias centradas no estudante favorecem não apenas a retenção de conteúdo, mas também o desenvolvimento do pensamento crítico. No mesmo sentido, \cite{irabedra2025active} e \cite{bonetti2025example} demonstram que o uso de projetos práticos e exemplos reais aumenta o engajamento discente, mitigando a abstração excessiva de conceitos técnicos.
ESE, \cite{mendez2024handbook} e \cite{avgeriou2024designing} discutem a importância de estruturar currículos que combinem teoria estatística com a execução prática, preparando o estudante para uma indústria orientada por evidências.

Trabalhos recentes têm investigado a aplicação dessas estratégias no ensino de ESE. \cite{meireles2024experience} analisaram a percepção de estudantes de graduação e pós-graduação sobre o uso de aprendizagem ativa, focando majoritariamente em atividades de Mapeamento Sistemático. Inspirados por essa abordagem, \cite{greco2025learning} adaptaram a metodologia especificamente para o contexto de graduação, deslocando o foco para a execução de estudos primários, como experimentos, estudos de caso e surveys. Em sua análise, \cite{greco2025learning} validaram a eficácia da abordagem pedagógica, demonstrando que 88,2\% dos estudantes relataram maior segurança ao final do curso e que a motivação apresentou correlação com a percepção de valor da disciplina.

Em contraste com esses estudos, este trabalho, embora situado no mesmo contexto de ensino prático explorado por \cite{greco2025learning}, avança a investigação ao deslocar o foco da validação pedagógica geral para uma análise psicométrica da autoeficácia percebida, entendida como a crença do estudante em sua capacidade de planejar, conduzir e analisar estudos empíricos. Enquanto os trabalhos anteriores avaliaram predominantemente satisfação e desempenho, esta pesquisa investiga estatisticamente como a motivação em tópicos específicos, como análise qualitativa e estatística, influencia a construção da confiança discente. Diferentemente de uma replicação direta, este estudo utiliza o mesmo contexto pedagógico como base para investigar um fenômeno distinto, buscando compreender os fatores que levam o estudante a se sentir apto a aplicar ESE na prática profissional.

\section{Metodologia}

Esse trabalho descreve uma experiência de ensino em uma turma da disciplina eletiva de ESE de 64 horas, ministrada no campus de uma universidade, para cursos de graduação. O objetivo geral da disciplina é capacitar o estudante nos fundamentos de ESE, abrangendo estudos primários e secundários. %A ementa inclui: conceituação e esclarecimento sobre experimentos controlados, estudos de caso e surveys; o processo de desenvolvimento de um projeto de pesquisa (incluindo atividades, formulação de questões de pesquisa, construção de teoria e análise de dados qualitativa/quantitativa); investigação de experimentos em engenharia de software; e prática supervisionada através de um experimento de engenharia de software em pequena escala.%

Durante a disciplina, a professora responsável e três monitores realizaram o acompanhamento das atividades. A turma foi composta por 63 estudantes, sendo 39 do curso de Engenharia de Software e 24 de Ciência da Computação. A sequência das atividades ao longo do semestre foi a seguinte (detalhes em \cite{greco2025learning}):

\begin{itemize}
    \item \textbf{Estudos primários e secundários:} aulas dialogadas \cite{wohlin2012experimentation, kitchenham2004procedures};
    \item \textbf{Experimentos controlados:} estudo dirigido em grupos de até 3, com seminários (Cap. 6--9 de \cite{wohlin2012experimentation});
    \item \textbf{Análise estatística:} aula expositiva + laboratório com JASP\footnote{JASP: https://jasp-stats.org/}, usando dados de \cite{fonseca2024avaliaccao};
    \item \textbf{TP1:} reanálise de experimento controlado em grupos de 3, com apresentação em sala;
    \item \textbf{Estudos de caso e \textit{surveys}:} estudos dirigidos (Cap. 5 de \cite{wohlin2012experimentation}, Cap. 3 de \cite{shull2008guide});
    \item \textbf{Análise qualitativa:} aula + laboratório com Taguette\footnote{Taguette: https://www.taguette.org/} e Flourish\footnote{Flourish: https://flourish.studio/} \cite{desiderio2024ready};
    \item \textbf{TP2:} planejamento, execução e análise de estudo primário (experimento, caso ou \textit{survey}) em grupos de até 6, com duas apresentações.
\end{itemize}

A metodologia adotada foi inspirada em \cite{meireles2024experience}, porém ajustada às características do perfil de estudantes de graduação da turma. As aulas introdutórias sobre estudos primários e secundários foram alinhadas ao nível da turma, e as sessões práticas de laboratório em análise estatística e qualitativa utilizaram JASP, Taguette e Flourish. Em função de limitações de espaço físico e do tamanho da turma, debates em grande grupo não foram priorizados. O tema Design Science Research não foi abordado, uma vez que a disciplina não tinha como público-alvo estudantes de pós-graduação. Diferentemente do que é descrito nos trabalhos práticos de \cite{meireles2024experience}, que enfatizava etapas de mapeamento sistemático e o planejamento de novos experimentos, nesta experiência concentrou-se o esforço em atividades práticas com estudos primários, mantendo a reanálise de um experimento controlado e adicionando a execução completa de um estudo primário (experimento, estudo de caso ou \textit{survey}) como eixo central de aprendizagem.

\subsection{Pesquisa de Percepção dos Estudantes}
A percepção dos estudantes foi investigada por meio de questionário online (Google Forms), estruturado a partir do instrumento de \cite{meireles2024experience}, com 31 questões (24 fechadas e 7 abertas) em cinco blocos: dados demográficos, percepção de aprendizagem antes/depois, motivação e valor atribuído. A participação foi voluntária, com TCLE assegurando anonimato e uso acadêmico. O questionário foi divulgado em sala ao final da disciplina; dos 63 matriculados, 21 responderam. O questionário completo e os procedimentos de análise estão disponíveis no repositório.

Para o tratamento dos dados, foi adotada uma abordagem mista (\textit{mixed-methods}), combinando análises quantitativas e qualitativas. Os dados quantitativos foram organizados, analisados e visualizados por meio de notebooks Python, com o uso de bibliotecas amplamente empregadas em análise de dados e estatística, como Pandas\footnote{Pandas: https://pandas.pydata.org/} para manipulação e organização dos dados, Scipy\footnote{Scipy: https://scipy.org/} para análises estatísticas e Matplotlib\footnote{Matplotlib: https://matplotlib.org/} para a geração dos gráficos. A implementação e execução dos scripts ocorreram na IDE Cursor\footnote{Cursor: https://www.cursor.so/}, utilizando notebooks Jupyter\footnote{Jupyter: https://jupyter.org/}, que permitiram a execução interativa das análises e a visualização dos resultados.

A análise quantitativa contemplou estatísticas descritivas e visualizações por meio de boxplots, com o objetivo de examinar o desempenho dos estudantes nas atividades propostas. Adicionalmente, foram conduzidas análises de regressão para investigar a relação entre variáveis associadas à motivação, conhecimento prévio e confiança percebida, bem como análises de confiabilidade para avaliar a consistência interna dos blocos do instrumento de coleta. Os procedimentos estatísticos foram realizados de forma colaborativa por dois pesquisadores, visando maior rigor na condução e interpretação dos resultados.

Para a análise qualitativa, as respostas às sete questões abertas do questionário foram submetidas a um processo de codificação aberta \cite{saldana2021coding}, no qual dois pesquisadores identificaram, de forma independente, códigos temáticos emergentes a partir dos relatos dos estudantes. Os códigos foram posteriormente consolidados por consenso e organizados em categorias temáticas, permitindo a triangulação com os resultados quantitativos e uma compreensão mais aprofundada da experiência vivenciada pelos discentes.

\section{Resultados}
Esta seção apresenta os resultados obtidos a partir dos dados coletados por meio do questionário e das planilhas de acompanhamento de notas/atividades. Os dados quantitativos são apresentados utilizando representações gráficas e análises estatísticas, enquanto os dados qualitativos são apresentados por meio de códigos temáticos emergentes e citações representativas dos estudantes.

\subsection{Desempenho da turma}
A Figura \ref{fig:boxplotsNotas} apresenta a distribuição das notas dos trabalhos práticos TP1 e TP2. Em ambos, a maior parte das notas concentra-se entre 7,5 e 9,5, indicando desempenho geral elevado da turma.

Observa-se que a mediana do TP2 é ligeiramente superior à do TP1, sugerindo melhora no desempenho ao longo da disciplina. O TP1 apresenta maior variabilidade, enquanto o TP2 mostra distribuição mais concentrada, com menor dispersão. Em ambos os trabalhos, há outliers associados a notas muito baixas, representando casos isolados de baixo desempenho.

De modo geral, os resultados indicam uma progressão no desempenho entre as duas avaliações e um nível de desempenho consistente para a maioria dos estudantes.

\begin{figure}[H]
    \centering
    \includegraphics[width=0.5\linewidth]{imgs/boxplot_notas.png}
    \caption{Performance dos estudantes nos trabalho práticos}
    \label{fig:boxplotsNotas}
\end{figure}

\subsection{Percepção de motivação}
A Figura \ref{fig:boxplots} apresenta uma análise comparativa do nível de conhecimento dos estudantes em sete tópicos de metodologia de pesquisa experimental, avaliados antes e depois da disciplina, por meio de gráficos de violino com boxplots internos e pontos individuais sobrepostos.

Cada subgráfico corresponde a um tópico específico — Estudos Experimentais, Experimento Controlado, Estudo de Caso, Survey, Análise Estatística, Análise Qualitativa e Revisão Sistemática da Literatura (RSL) — comparando as condições \textbf{“Antes”} e \textbf{“Depois”}. A escala de resposta varia de 0 a 5, representando o nível de conhecimento autoavaliado.

A visualização permite observar mudanças nas medianas, na dispersão e na forma das distribuições entre as duas condições, bem como a presença de valores atípicos, possibilitando uma avaliação visual do impacto da disciplina sobre o conhecimento dos estudantes em cada tópico.

\begin{figure}[H]
    \centering
    \includegraphics[width=1\linewidth]{imgs/boxplots.png}
    \caption{Distribuição dos niveis de conhecimento antes e depois do curso através de sete tópicos}
    \label{fig:boxplots}
\end{figure}

\subsection{Análise Qualitativa das Percepções Discentes}

Para compreender como a autoeficácia foi construída ao longo da disciplina e quais aspectos da vivência prática mais influenciaram a percepção dos estudantes, realizou-se uma análise qualitativa das respostas às sete questões abertas do questionário de percepção, por meio de codificação aberta. Os resultados são organizados em quatro eixos temáticos.

\subsubsection{Experiência prévia e ponto de partida}

Dos 21 respondentes, 13 declararam não possuir experiência prévia com ESE. Relatos como \textit{``Nunca conduzi um estudo''} (E1), \textit{``Nenhum''} (E4, E12, entre outros) e \textit{``Não utilizei nenhum, foi a primeira vez que fui apresentado à engenharia de software experimental''} (E17) evidenciam que a disciplina representou, para a maioria, o primeiro contato estruturado com métodos empíricos. Apenas quatro relataram experiência prévia diversificada; os demais mencionaram experiências pontuais. A autoeficácia ao final do curso foi construída predominantemente durante a disciplina.

\subsubsection{Construção da autoeficácia}

A análise das justificativas dos estudantes sobre sua capacidade percebida ao final da disciplina revelou três códigos temáticos centrais.

O código mais frequente, presente em 12 dos 21 relatos, foi a \textbf{autoeficácia construída pela vivência prática}. Os estudantes atribuíram a confiança à experiência de planejar, executar e analisar um estudo empírico:

\begin{itemize}
    \item E1: \textit{``Após a disciplina eu consigo mapear e estruturar um estudo, consigo fazer o passo a passo até o levantamento e estudo dos dados.''}
    \item E3: \textit{``Com o trabalho prático, onde tivemos que elaborar e dirigir um estudo de caso, me fez me sentir mais confiante para realizar pesquisas, já que agora possuo essa experiência.''}
    \item E17: \textit{``Agora tenho uma boa noção de como realizar análises qualitativas e quantitativas e me sinto bem mais confiante em conduzir qualquer tipo de experimento.''}
\end{itemize}

O segundo código, a \textbf{clareza metodológica adquirida}, apareceu em 6 relatos: \textit{``O fato da disciplina ter mostrado o passo a passo de como conduzir uma pesquisa me fez sentir menos insegurança''} (E4); \textit{``Os passos para uma análise quantitativa e/ou qualitativa são bem definidos, eu não tinha essa percepção antes''} (E20).

Por fim, dois relatos indicaram \textbf{dificuldades remanescentes}, sugerindo que a construção da autoeficácia não foi uniforme entre todos os participantes: \textit{``O projeto final me definiu uma nova visão sobre condução de uma pesquisa, mas ainda é complicado, mas nada que mais estudo não solucione''} (E14).

\subsubsection{Impacto dos estudos dirigidos}

Os estudos dirigidos --- atividades conduzidas de forma autônoma pelos alunos, organizadas em grupos e apresentadas periodicamente para a turma --- foram avaliados positivamente pela maioria dos respondentes, embora desafios relevantes tenham sido identificados.

O principal facilitador foi o mecanismo de \textbf{aprender ensinando}: \textit{``Acredito que ler e ter que interpretar para entender o conteúdo [...] fez com que a gente se sentisse mais capaz (autoconfiança) e trabalhasse o raciocínio, já que de certa forma precisávamos elaborar `uma aula' para apresentá-la posteriormente''} (E3). A \textbf{autonomia} e o \textbf{trabalho colaborativo} foram igualmente valorizados: \textit{``Essa metodologia me possibilitou mais autonomia e me fez aprender bastante''} (E8); \textit{``O fato de ter sido em grupo facilitou a compreensão dos conteúdos''} (E4).

Entre os desafios, a \textbf{fragmentação do conteúdo em grupos} foi a dificuldade mais expressiva: \textit{``Facilitou para absorver melhor os conteúdos pelos quais eu ficava responsável, mas dificultou em ter uma visão do conteúdo todo. Como os capítulos eram muito grandes e o tempo era muito curto, meu grupo acabou dividindo os conteúdos entre os integrantes [...] acabou que eu não absorvi bem os conteúdos que foram resumidos pelas outras pessoas''} (E7). A barreira linguística dos materiais em inglês também foi mencionada como fator dificultador (E14).

\subsubsection{Fatores motivacionais e contribuição para a formação}

A análise das respostas sobre motivação revelou cinco códigos temáticos que explicam qualitativamente por que a motivação é o principal preditor da confiança, conforme evidenciado na análise de regressão (Seção \ref{sec:regressao}).

A \textbf{aplicação prática dos conteúdos} foi o fator motivacional mais citado (8 estudantes): \textit{``As aplicações práticas das técnicas passadas ao longo da cadeira''} (E2); \textit{``A possibilidade de planejar, conduzir e analisar os dados, na prática, me motivou a participar ativamente das atividades''} (E13). A \textbf{ausência de provas tradicionais} emergiu como diferencial relevante: \textit{``A não necessidade de provas ajudou bem mais no aprendizado''} (E6); \textit{``A forma como a disciplina foi conduzida com estudos dirigidos tirando da prática tradicional das avaliações teve impacto positivo no aprendizado''} (E8). A \textbf{conexão com projetos pessoais}, especialmente o TCC, também se mostrou relevante: \textit{``Os conteúdos abordados na disciplina me ajudaram a construir várias ideias para o meu TCC''} (E17); \textit{``O artigo que eu estava fazendo em paralelo com a disciplina me ajudou e me orientou muito''} (E10).

Quanto à contribuição para a formação, 15 dos 21 respondentes mencionaram a preparação para conduzir pesquisas futuras: \textit{``Posso conduzir experimentos com um melhor direcionamento [...]. Assim posso extrair resultados significativos com questões que venham a surgir na minha vida profissional''} (E7); \textit{``A disciplina fez com que eu tivesse mais confiança em relação ao que posso fazer em uma pesquisa''} (E3). Uma sugestão identificada foi evitar a fusão tardia de equipes para o TP2: \textit{``Talvez não unir dois grupos para o trabalho prático 2, às vezes a dinâmica com outra equipe, introduzida de maneira tardia, pode não ser tão interessante''} (E17).

\subsection{Análise de Regressão} \label{sec:regressao}
Para identificar quais fatores influenciam a confiança dos estudantes em conduzir pesquisas experimentais, foi ajustado um modelo de Random Forest Regressor. A Figura \ref{fig:importanciaIndividual} apresenta a importância relativa das variáveis no modelo, evidenciando que \textbf{a motivação média dos estudantes} é, de longe, o fator mais determinante da confiança (41,1\%), seguida da \textbf{motivação para Análise Qualitativa} (20,1\%). Em posição secundária, Engajamento (8,4\%), Conhecimento prévio (8,1\%) e Motivação para Survey (8,0\%) contribuem de forma moderada; Estudo de Caso (6,7\%) e RSL (4,7\%) possuem contribuição menor. As variáveis de menor impacto foram motivação para análise estatística (1,9\%) e curso de origem (0,9\%), indicando que a confiança é mais influenciada por aspectos motivacionais e pedagógicos do que por características demográficas.

 \begin{figure}[!ht]
     \centering
     \includegraphics[width=0.75\linewidth]{imgs/importancia_individual.png}
     \caption{Importância das variáveis na confiança do estudante}
     \label{fig:importanciaIndividual}
 \end{figure}

\subsection{Análise de Confiabilidade do Instrumento}

Para verificar a robustez do instrumento de coleta, realizou-se uma análise psicométrica baseada no $\alpha$ de Cronbach, nas correlações item-total e na simulação de exclusão de itens. A Figura \ref{fig:AlphaAnalisys} sintetiza os resultados para os três blocos do questionário.
\begin{figure}[H]
    \centering
    \includegraphics[width=1\linewidth]{cronbach_alpha_analysis.png}
    \caption{Alpha de Cronbach e correlações item-total por bloco}
    \label{fig:AlphaAnalisys}
\end{figure}

Os Blocos 1 (Conhecimento Antes, $\alpha = 0{,}928$) e 2 (Conhecimento Depois, $\alpha = 0{,}832$) apresentaram consistência interna acima do limiar recomendado ($\alpha \geq 0{,}70$), com correlações item-total entre 0,494 e 0,888, indicando forte alinhamento dos itens aos constructos medidos. O Bloco 3 (Motivação), por outro lado, obteve $\alpha = 0{,}616$, com correlações item-total próximas ao limiar mínimo de aceitabilidade (0,323--0,472).

A análise de exclusão de itens revelou que, nos Blocos 1 e 2, a remoção de qualquer item produziria variação marginal no $\alpha$. No Bloco 3, a exclusão do item de RSL elevaria o $\alpha$ para 0,640, enquanto a remoção de ``Estudo de Caso'' o reduziria para 0,520, evidenciando heterogeneidade no constructo de motivação. Esses resultados sugerem a necessidade de revisão dos itens de motivação em futuras aplicações do instrumento.



\section{Lições Aprendidas}

A implementação da Aprendizagem Ativa em uma turma de 63 estudantes gerou aprendizados relevantes para futuras ofertas da disciplina. Esta seção apresenta as perspectivas da equipe docente e dos monitores, articulando-as com os achados qualitativos e quantitativos apresentados anteriormente.

\subsection{O que funcionou}

A progressão entre os dois trabalhos práticos mostrou-se um dos elementos mais eficazes. No TP1, os estudantes reanalisaram um experimento controlado publicado; no TP2, planejaram, executaram e analisaram seus próprios estudos primários. Essa transição gradual --- da reanálise à condução autônoma --- foi percebida pela professora como fundamental para a autoeficácia, corroborada pelos dados: a mediana do TP2 foi superior à do TP1. Os estudos dirigidos promoveram o mecanismo de \textit{aprender ensinando}; a monitoria observou discussões espontâneas nos seminários. As sessões de laboratório (JASP, Taguette, Flourish) conectaram teoria e prática; a presença de três monitores \cite{greco2025learning} foi importante para atender a turma. A ausência de provas tradicionais foi citada pelos estudantes como fator motivacional e refletiu-se em maior engajamento contínuo.

\subsection{O que não funcionou ou pode ser aprimorado}

A principal dificuldade observada pelos monitores foi a \textbf{curva de aprendizado em análise estatística}. No TP1, alguns grupos não conseguiram reproduzir os testes estatísticos ou não compreenderam a necessidade de verificar a normalidade dos dados antes de selecionar o teste de hipóteses adequado. Esse desafio sugere a necessidade de sessões adicionais de laboratório focadas exclusivamente em análise estatística antes da atribuição do TP1.

A \textbf{fragmentação do conteúdo nos estudos dirigidos} foi uma dificuldade identificada tanto nos relatos qualitativos quanto nas observações dos monitores. Quando confrontados com capítulos extensos e prazos curtos, alguns grupos dividiram o material entre os integrantes, resultando em compreensão parcial do conteúdo. Uma possível melhoria seria reduzir o escopo dos estudos dirigidos ou adicionar sessões de discussão pós-seminário em formato de mesa-redonda, criando um espaço para esclarecimento coletivo de dúvidas.

A \textbf{barreira linguística} com materiais em inglês também foi mencionada como fator dificultador. Embora a leitura em inglês seja uma competência esperada na formação acadêmica, a equipe docente reconhece que a disponibilização de materiais complementares em português pode facilitar a compreensão inicial dos conceitos.

A \textbf{fusão tardia de equipes} para o TP2 foi reportada por um estudante como problemática, indicando que a dinâmica de grupo já estabelecida pode ser prejudicada pela introdução de novos integrantes em estágio avançado do semestre.

Por fim, os monitores observaram que, no início do semestre, \textbf{alguns estudantes não compreenderam que deveriam completar todos os estudos dirigidos}, e não apenas aqueles que iriam apresentar. Essa observação reforça a importância de comunicar explicitamente as expectativas de cada atividade desde o primeiro dia de aula.

\subsection{Para quem a abordagem funcionou melhor}

A análise de regressão (Seção \ref{sec:regressao}) evidenciou que o curso de origem do estudante (Engenharia de Software ou Ciência da Computação) teve impacto quase nulo (0,9\%) na confiança adquirida, indicando que a abordagem é igualmente eficaz independentemente da formação base. Os estudantes que mais se beneficiaram foram aqueles que perceberam conexão direta entre a disciplina e seus projetos pessoais --- especialmente o TCC ---, conforme evidenciado nos relatos qualitativos. Estudantes sem qualquer experiência prévia em ESE (13 dos 21 respondentes) demonstraram os maiores ganhos de autoeficácia, sugerindo que a disciplina é particularmente valiosa para alunos em primeiro contato com métodos empíricos.

\section{Considerações finais}
Os resultados indicam que estratégias de aprendizagem ativa contribuem para o desenvolvimento do conhecimento metodológico e para o fortalecimento da autoconfiança dos estudantes em Engenharia de Software Experimental, sendo a motivação o fator mais fortemente associado à confiança para conduzir pesquisas empíricas. A análise qualitativa revelou que a autoeficácia foi construída predominantemente pela vivência prática de condução de estudos completos, com 12 dos 21 respondentes atribuindo sua confiança diretamente a essa experiência. A triangulação entre os dados quantitativos e qualitativos evidencia que a motivação, identificada como principal preditor da confiança no modelo de regressão, é alimentada pela aplicação prática dos conteúdos, pela ausência de provas tradicionais e pela conexão com projetos pessoais dos estudantes.

A experiência também evidenciou desafios concretos: a curva de aprendizado em análise estatística, a fragmentação do conteúdo nos estudos dirigidos e a barreira linguística. Para docentes que considerem adotar abordagem semelhante, recomenda-se a progressão gradual entre atividades de reanálise e condução autônoma de estudos, a comunicação explícita das expectativas desde o início do semestre, e a alocação de monitores em número proporcional ao tamanho da turma.

Como perspectivas de trabalhos futuros, destacam-se o aprimoramento do instrumento de coleta, especialmente a revisão dos itens de motivação cuja confiabilidade foi questionável ($\alpha = 0{,}616$), bem como a replicação do estudo em outras turmas e contextos institucionais. Investigações com amostras maiores e análises longitudinais podem oferecer evidências mais robustas sobre o impacto dessas estratégias na formação de profissionais orientados por evidências.

\section*{Artefatos relacionados ao trabalho}
\label{sec:artefatos}
Os artefatos associados a este estudo, incluindo os instrumentos de coleta, os dados coletados e os \textit{notebooks} utilizados nas análises, estão disponíveis publicamente em um repositório da Anonymous Github: \url{https://anonymous.4open.science/r/ees-feedback-F30B}.

\section*{Agradecimentos}

\subsection*{Uso de Inteligência Artificial}
 Neste trabalho, foram utilizadas ferramentas de Inteligência Artificial Generativa exclusivamente para suporte técnico na elaboração de gráficos incluídos na seção de resultados. Para isso, empregou-se o Cursor AI (agente assistente de código), utilizado apenas para auxiliar na depuração do código empregado para produzir os gráficos apresentados no artigo.

Nenhuma ferramenta de IA foi utilizada para redigir, revisar ou estruturar o texto do artigo, tampouco para elaborar citações, interpretações ou discussões. As decisões metodológicas, análises, resultados e conclusões são integralmente de responsabilidade das pessoas autoras.

\bibliographystyle{sbc}
\bibliography{sbc-template}

\end{document}
